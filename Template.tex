%\documentclass[spanish,10pt]{beamer}
\documentclass[spanish]{beamer}
% Class options include: notes, notesonly, handout, trans,
    %                        hidesubsections, shadesubsections,
    %                        inrow, blue, red, grey, brown
    \mode<presentation>
{
    \usetheme{Warsaw}
    \usecolortheme{rose}
    %\usecolortheme{dolphin}
    \setbeamertemplate{itemize items}[circle] % if you want a ball
    \setbeamertemplate{itemize subitem}[triangle] % if you wnat a circle
    \setbeamertemplate{itemize subsubitem}[ball] % if you want a triangle
    \setbeamertemplate{sections/subsections in toc}[circle]
    \setbeamertemplate{enumerate items}[circle]
    \setbeamercovered{transparent}
    % or whatever (possibly just delete it)
}
% Theme for beamer presentation.
\usepackage[spanish]{babel}
\usepackage[utf8]{inputenc}
%\usepackage{beamerthemesplit}

\mode<presentation>{
\usebackgroundtemplate{\includegraphics[width=\paperwidth]{gul_logo_b.png}}
} 

\usefonttheme{professionalfonts}% font de LaTeX
\usetheme{Warsaw}
% Tema escogido en este ejemplo
%\setBeamercovered{transparent}
%%%% packages y comandos personales %%%%
\usepackage[utf8]{inputenc}
\usepackage{latexsym} % Simbolos
\newtheorem{Ejemplo}{Ejemplo}
\newtheorem{Definicion}{Definicion}
\title[Git y Mercurial vs. SVN - GUL-UCM]
{Git y Mercurial vs. Subversion
}
\author[\url{http://gulucm.org}]
{
Federico Mon \\
Nehuén Eloy Benítez
}
%{GUL-UCM} \\
%{\url{http://gulucm.org}}\\
%\vspace*{0.5cm}}
\institute{Facultad de Informática.\\Universidad Complutense de Madrid.}
\date{
I Jornadas de Software Libre - GUL-UCM\\
Marzo 2011}

\AtBeginSection[]
{
%%  \logo{\vbox{\vskip0.3cm\hbox{\includegraphics[width=2cm]{gulucm.eps}}}}
    \begin{frame}<beamer>
    \tableofcontents[currentsection]
    \end{frame}
\logo{}
}

%%\logo{\vbox{\vskip0.3cm\hbox{\includegraphics[width=2cm]{gulucm.eps}}}}

\newcommand*\oldmacro{}%
\let\oldmacro\insertshorttitle%
\renewcommand*\insertshorttitle{%
  \oldmacro\hfill%
  \insertframenumber\,/\,\inserttotalframenumber}



\begin{document}
    \frame{\titlepage}
    \section{Control de versiones}
        \begin{frame}{Las versiones de un fichero}
        Qué es una versión de algo?
        Todos cuando trabajamos con un ordenador en un proyecto, llegamos a un
        punto en el cual sucede un hito.
        Este hito depende de en qué estemos trabajando: pero puede ser que 
        tengamos un documento aceptable, o que hayamos solucionado un problema 
        parcialmente... la cuestión es que queremos guardar nuestro trabajo, 
        pero debemos seguir trabajando de todas maneras.
        
        Normalmente con un ordenador, lo que hacemos es guardar el fichero con 
        un nombre y ponerle un número, y seguimos trabajando sobre una copia de
        este fichero.
        
        El problema es que cuando el número de versiones empieza a crecer, es 
        muy difícil llevar manualmente un seguimiento de las versiones, volver 
        a la que queremos, etc.
        
        Esta tarea se complica mucho más cuando son varias personas las que 
        trabajan en un proyecto en paralelo, y además juntar los cambios hechos 
        es algo inevitable.
        
        \begin{frame}{Sistemas de Control de Versiones}
        
        Los sistemas de control de versiones hacen más fácil controlar las 
        versiones de nuestro trabajo.
        
        Permiten por ejemplo, hacer un comentario para cada versión que queremos
         guardar, llevan un control de "quién ha hecho qué", permiten volver 
        atrás en las versiones, y facilitan que todas las personas involucradas 
        puedan acceder siempre a la última versión del trabajo, así como ver el 
        historial de las versiones, o las diferencias entre éstas.
        
        Existen dos tipos de sistemas de control de versiones, los centralizados
        , donde toda la información es almacenada en un servidor central, 
        llamado repositorio, y los distribuidos, donde esta información se 
        replica también en los clientes.
        
        La ventaja más clara de los sistemas distribuidos frente a los 
        centralizados es que no es necesario poder conectar con el servidor 
        central para cada acción que queremos realizar.
        
        \begin{frame}{CVCS}
        
        
        
        
        
        SUBVERSION
            Crear el repositorio:
               $ svnadmin create /svnrepos
               $ vim /svnrepos/conf/svnserve.conf
                anon-access = none // read
                auth-access = write
                password-db = passwd
               $ vim svnrepos/conf/passwd
                tony = mypassword
               $ svn import /projects/myproject file:///svnrepos/myproject
               $ svnserve -d
               $ svn co svn://localhost/svnrepos/myyrailsproject

        
               
        
        
%            \begin{itemize}
%                \item Asociación de estudiantes de la Universidad Complutense.
%                \item Sede en la Facultad de Informática.
%                \item Dedicada a promover el uso de Software Libre.
%            \end{itemize}
        \end{frame}
        \begin{frame}{}
            Se trata de una filosofía que queda resumida y representada en 4 
            libertades:
            \begin{enumerate}
            \setcounter{enumi}{-1}
                \item La libertad de usar el programa, con cualquier propósito.
                \item La libertad de estudiar \footnote{Las libertades 1 y 3 
                requieren acceso al código fuente porque estudiar y modificar 
                software sin su código fuente es muy poco viable} cómo funciona 
                el programa y modificarlo, adaptándolo a tus necesidades.
                \item La libertad de distribuir copias del programa, con lo cual
                 puedes ayudar a tu prójimo.
                \item La libertad de mejorar el programa y hacer públicas esas 
                mejoras a los demás, de modo que toda la comunidad se beneficie.
            \end{enumerate}
        \end{frame}

    \section{Quiénes somos}
        \begin{frame}{Representación}
            \begin{description}[Left]
                \item [Presidente:] Federico G. Mon Trotti
                \item [Vicepresidente:] Nehuén Eloy Benítez
                \item [Secretario:] Sergio D'Antonio
                \item [Tesorero:] Luis San Juan
                \item [Vocal de representación estudiantil:] Jorge Maestre Vidal
                \item [Vocal de actividades:] Salvador de la Puente González
            \end{description}
        \end{frame}
        \begin{frame}{Recursos y contacto}
            (link a estatutos, lista de correo, pagina web, etc)
        \end{frame}
\end{document}
