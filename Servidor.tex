%\documentclass[spanish,10pt]{beamer}
\documentclass[spanish]{beamer}
% Class options include: notes, notesonly, handout, trans,
    %                        hidesubsections, shadesubsections,
    %                        inrow, blue, red, grey, brown
    \mode<presentation>
{
    \usetheme{Warsaw}
    \usecolortheme{rose}
    %\usecolortheme{dolphin}
    \setbeamertemplate{itemize items}[circle] % if you want a ball
    \setbeamertemplate{itemize subitem}[triangle] % if you wnat a circle
    \setbeamertemplate{itemize subsubitem}[ball] % if you want a triangle
    \setbeamertemplate{sections/subsections in toc}[circle]
    \setbeamertemplate{enumerate items}[circle]
    \setbeamercovered{transparent}
    % or whatever (possibly just delete it)
}
% Theme for beamer presentation.
\usepackage[spanish]{babel}
\usepackage[utf8]{inputenc}
%\usepackage{beamerthemesplit}

\mode<presentation>{
\usebackgroundtemplate{\includegraphics[width=\paperwidth]{gul_logo_b.png}}
} 

\usefonttheme{professionalfonts}% font de LaTeX
\usetheme{Warsaw}
% Tema escogido en este ejemplo
%\setBeamercovered{transparent}
%%%% packages y comandos personales %%%%
\usepackage[utf8]{inputenc}
\usepackage{latexsym} % Simbolos
\newtheorem{Ejemplo}{Ejemplo}
\newtheorem{Definicion}{Definicion}
\title[Monta tu propio servidor - GUL-UCM]
{Monta tu propio servidor
}
\author[\url{http://gulucm.org}]
{
Federico Mon\\
Salvador de la Puente
}
%{GUL-UCM} \\
%{\url{http://gulucm.org}}\\
%\vspace*{0.5cm}}
\institute{Facultad de Informática.\\Universidad Complutense de Madrid.}
\date{
I Jornadas de Software Libre - GUL-UCM\\
Marzo 2011}

\AtBeginSection[]
{
%%  \logo{\vbox{\vskip0.3cm\hbox{\includegraphics[width=2cm]{gulucm.eps}}}}
    \begin{frame}<beamer>
    \tableofcontents[currentsection]
    \end{frame}
\logo{}
}

%%\logo{\vbox{\vskip0.3cm\hbox{\includegraphics[width=2cm]{gulucm.eps}}}}

\newcommand*\oldmacro{}%
\let\oldmacro\insertshorttitle%
\renewcommand*\insertshorttitle{%
  \oldmacro\hfill%
  \insertframenumber\,/\,\inserttotalframenumber}



\begin{document}
    \frame{\titlepage}
\begin{frame}{Que es un servidor}
\end{frame}
\begin{frame}{Para que queremos un servidor}
\end{frame}
\begin{frame}{Nota sobre seguridad}
\end{frame}
\begin{frame}{Que son las DNS}
\end{frame}
\begin{frame}{Dyndns/freedns}
\begin{itemize}
\item Necesitamos algo que actualice las DNS con nuestras cambiantes IP dinámicas.
\item DynDNS y FreeDNS son servicios que nos permiten hacer esto
\url{http://dyndns.com}
\end{itemize}
\end{frame}
\begin{frame}{LAMP}
\end{frame}
\begin{frame}{Linux}
\end{frame}
\begin{frame}{Apache}
\end{frame}
\begin{frame}{Mysql}
\end{frame}
\begin{frame}{Php}
\end{frame}
\begin{frame}{Phpmyadmin}
\end{frame}
\begin{frame}{Drupal}
\end{frame}
\begin{frame}{Dokuwiki}
\end{frame}
\begin{frame}{FTP/Filezilla}
\end{frame}
\begin{frame}{SSH}
\end{frame}
\begin{frame}{SFTP/FireFTP}
\end{frame}
\begin{frame}{Samba}
\end{frame}
\begin{frame}{Plugins Drupal/Dokuwiki}
\end{frame}
\begin{frame}{Dokuwiki - ODT, PDF}
\end{frame}
\begin{frame}{Drupal - FCKEditor, SPAM}
\end{frame}
\begin{frame}{Banshee}
\end{frame}
\begin{frame}{NFS}
\end{frame}

        
        
%            \begin{itemize}
%                \item Asociación de estudiantes de la Universidad Complutense.
%                \item Sede en la Facultad de Informática.
%                \item Dedicada a promover el uso de Software Libre.
%            \end{itemize}
        \end{frame}
        \begin{frame}{}
            Se trata de una filosofía que queda resumida y representada en 4 
            libertades:
            \begin{enumerate}
            \setcounter{enumi}{-1}
                \item La libertad de usar el programa, con cualquier propósito.
                \item La libertad de estudiar \footnote{Las libertades 1 y 3 
                requieren acceso al código fuente porque estudiar y modificar 
                software sin su código fuente es muy poco viable} cómo funciona 
                el programa y modificarlo, adaptándolo a tus necesidades.
                \item La libertad de distribuir copias del programa, con lo cual
                 puedes ayudar a tu prójimo.
                \item La libertad de mejorar el programa y hacer públicas esas 
                mejoras a los demás, de modo que toda la comunidad se beneficie.
            \end{enumerate}
        \end{frame}

    \section{Quiénes somos}
        \begin{frame}{Representación}
            \begin{description}[Left]
                \item [Presidente:] Federico G. Mon Trotti
                \item [Vicepresidente:] Nehuén Eloy Benítez
                \item [Secretario:] Sergio D'Antonio
                \item [Tesorero:] Luis San Juan
                \item [Vocal de representación estudiantil:] Jorge Maestre Vidal
                \item [Vocal de actividades:] Salvador de la Puente González
            \end{description}
        \end{frame}
        \begin{frame}{Recursos y contacto}
            (link a estatutos, lista de correo, pagina web, etc)
        \end{frame}
\end{document}
