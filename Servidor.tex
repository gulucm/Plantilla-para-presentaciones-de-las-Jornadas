%\documentclass[spanish,10pt]{beamer}
\documentclass[spanish]{beamer}
% Class options include: notes, notesonly, handout, trans,
    %                        hidesubsections, shadesubsections,
    %                        inrow, blue, red, grey, brown
    \mode<presentation>
{
    \usetheme{Warsaw}
    \usecolortheme{rose}
    %\usecolortheme{dolphin}
    \setbeamertemplate{itemize items}[circle] % if you want a ball
    \setbeamertemplate{itemize subitem}[triangle] % if you wnat a circle
    \setbeamertemplate{itemize subsubitem}[ball] % if you want a triangle
    \setbeamertemplate{sections/subsections in toc}[circle]
    \setbeamertemplate{enumerate items}[circle]
    \setbeamercovered{transparent}
    % or whatever (possibly just delete it)
}
% Theme for beamer presentation.
\usepackage[spanish]{babel}
\usepackage[utf8]{inputenc}
%\usepackage{beamerthemesplit}

\mode<presentation>{
\usebackgroundtemplate{\includegraphics[width=\paperwidth]{gul_logo_b.png}}
} 

\usefonttheme{professionalfonts}% font de LaTeX
\usetheme{Warsaw}
% Tema escogido en este ejemplo
%\setBeamercovered{transparent}
%%%% packages y comandos personales %%%%
\usepackage[utf8]{inputenc}
\usepackage{latexsym} % Simbolos
\newtheorem{Ejemplo}{Ejemplo}
\newtheorem{Definicion}{Definicion}
\title[Monta tu propio servidor - GUL-UCM]
{Monta tu propio servidor
}
\author[\url{http://gulucm.org}]
{
Federico Mon\\
Salvador de la Puente
}
%{GUL-UCM} \\
%{\url{http://gulucm.org}}\\
%\vspace*{0.5cm}}
\institute{Facultad de Informática.\\Universidad Complutense de Madrid.}
\date{
I Jornadas de Software Libre - GUL-UCM\\
Marzo 2011}

\AtBeginSection[]
{
%%  \logo{\vbox{\vskip0.3cm\hbox{\includegraphics[width=2cm]{gulucm.eps}}}}
    \begin{frame}<beamer>
    \tableofcontents[currentsection]
    \end{frame}
\logo{}
}

%%\logo{\vbox{\vskip0.3cm\hbox{\includegraphics[width=2cm]{gulucm.eps}}}}

\newcommand*\oldmacro{}%
\let\oldmacro\insertshorttitle%
\renewcommand*\insertshorttitle{%
  \oldmacro\hfill%
  \insertframenumber\,/\,\inserttotalframenumber}



\begin{document}
\frame{\titlepage}
\begin{frame}{Que es un servidor}
\end{frame}
\begin{frame}{Para que queremos un servidor}
\end{frame}
\begin{frame}{Nota sobre seguridad}
\end{frame}
\begin{frame}{Que son las DNS}
\end{frame}
\begin{frame}{Dyndns/freedns}
\begin{itemize}
\item Necesitamos algo que actualice las DNS con nuestras cambiantes IP 
dinámicas.
\item DynDNS y FreeDNS son servicios que nos permiten esto.
\item Nos creamos una cuenta en DynDNS, por ejemplo: \url{http://dyndns.com}
\item E instalamos \textbf{ddclient} en nuestro servidor
\item \textbf{ddclient} se encarga de actualizar los registros de DynDNS
cuando es necesario
\end{itemize}
\end{frame}
\begin{frame}{LAMP}
\begin{itemize}
\item \textbf{LAMP} son las siglas de \textbf{L}inux, \textbf{A}pache, \textbf{M}ySQL y \textbf{P}HP
\item Es una pila de tecnologías muy usadas conjuntamente para hacer sitios web.
\item También existen WAMP, MAMP, SAMP, OAMP y XAMPP
\item Se puede cambiar también PHP por Perl o Python
\item Además es lámpara en inglés
\end{itemize}
\end{frame}
\begin{frame}{Linux}
\begin{itemize}
\item Es el sistema operativo más usado para servidores web
\item Es libre, robusto y gratuito
\item Linux: \url{http://kernel.org}
\item GNU: \url{http://gnu.org}
\item GNU/Linux Debian: \url{http://debian.org}
\item GNU/Linux Ubuntu: \url{http://ubuntu.com}
\end{itemize}
\end{frame}
\begin{frame}{Apache}
\begin{itemize}
\item Es el servidor web más usado
\item Es libre, robusto y gratuito (A que parece Copy\&Paste?)
\item Es quizá la parte más importante del servidor
\item Apache: \url{http://apache.org}
\end{itemize}
\end{frame}
\begin{frame}{Mysql}
\begin{itemize}
\item Es el servidor de bases de datos libre más famoso
\item No hace falta que diga que es libre, gratuito y robusto, ¿verdad?
\item Es una parte esencial para tener muchos datos ordenadamente
\item Ha cambiado de manos recientemente, era de Sun y ahora de Oracle
\item MySQL: \url{http://mysql.org}
\end{itemize}
\end{frame}
\begin{frame}{Php}
\begin{itemize}
\item Es un lenguaje que permite hacer dinámicas las páginas web
\item Trabaja en conjunto con Apache insertando variables entre el HTML
\item Es uno de los lenguajes más usados en desarrollo web
\item Ha cambiado de manos recientemente, era de Sun y ahora de Oracle
\item PHP: \url{http://php.net}
\end{itemize}
\end{frame}
\begin{frame}{Phpmyadmin}
\end{frame}
\begin{frame}{Drupal}
\end{frame}
\begin{frame}{Dokuwiki}
\end{frame}
\begin{frame}{FTP/Filezilla}
\begin{itemize}
\item FTP significa File Transfer Protocol
\item Como bien se intuye, es un protocolo que permite transferir archivos
\item Donde más se usa es probablemente, para subir ``páginas web''
\item Necesitamos el demonio ftp en el servidor, y un cliente para conectarnos
\item Como cliente veremos Filezilla: \url{http://filezilla-project.org}
\end{itemize}
\end{frame}
\begin{frame}{SSH}
\begin{itemize}
\item SSH viene de Secure Shell
\item 
\item 
\end{itemize}
\end{frame}
\begin{frame}{SFTP/FireFTP}
\end{frame}
\begin{frame}{Samba}
\end{frame}
\begin{frame}{Plugins Drupal/Dokuwiki}
\end{frame}
\begin{frame}{Dokuwiki - ODT, PDF}
\end{frame}
\begin{frame}{Drupal - FCKEditor, SPAM}
\end{frame}
\begin{frame}{Banshee}
\end{frame}
\begin{frame}{NFS}
\end{frame}
\end{document}
