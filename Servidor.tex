%\documentclass[spanish,10pt]{beamer}
\documentclass[spanish]{beamer}
% Class options include: notes, notesonly, handout, trans,
    %                        hidesubsections, shadesubsections,
    %                        inrow, blue, red, grey, brown
    \mode<presentation>
{
    \usetheme{Warsaw}
    \usecolortheme{rose}
    %\usecolortheme{dolphin}
    \setbeamertemplate{itemize items}[circle] % if you want a ball
    \setbeamertemplate{itemize subitem}[triangle] % if you wnat a circle
    \setbeamertemplate{itemize subsubitem}[ball] % if you want a triangle
    \setbeamertemplate{sections/subsections in toc}[circle]
    \setbeamertemplate{enumerate items}[circle]
    \setbeamercovered{transparent}
    % or whatever (possibly just delete it)
}
% Theme for beamer presentation.
\usepackage[spanish]{babel}
\usepackage[utf8]{inputenc}
%\usepackage{beamerthemesplit}

\mode<presentation>{
\usebackgroundtemplate{\includegraphics[width=\paperwidth]{gul_logo_b.png}}
} 

\usefonttheme{professionalfonts}% font de LaTeX
\usetheme{Warsaw}
% Tema escogido en este ejemplo
%\setBeamercovered{transparent}
%%%% packages y comandos personales %%%%
\usepackage[utf8]{inputenc}
\usepackage{latexsym} % Simbolos
\newtheorem{Ejemplo}{Ejemplo}
\newtheorem{Definicion}{Definicion}
\title[Monta tu propio servidor - GUL-UCM]
{Monta tu propio servidor
}
\author[\url{http://gulucm.org}]
{
Federico Mon\\
Salvador de la Puente
}
%{GUL-UCM} \\
%{\url{http://gulucm.org}}\\
%\vspace*{0.5cm}}
\institute{Facultad de Informática.\\Universidad Complutense de Madrid.}
\date{
I Jornadas de Software Libre - GUL-UCM\\
Marzo 2011}

\AtBeginSection[]
{
%%  \logo{\vbox{\vskip0.3cm\hbox{\includegraphics[width=2cm]{gulucm.eps}}}}
    \begin{frame}<beamer>
    \tableofcontents[currentsection]
    \end{frame}
\logo{}
}

%%\logo{\vbox{\vskip0.3cm\hbox{\includegraphics[width=2cm]{gulucm.eps}}}}

\newcommand*\oldmacro{}%
\let\oldmacro\insertshorttitle%
\renewcommand*\insertshorttitle{%
  \oldmacro\hfill%
  \insertframenumber\,/\,\inserttotalframenumber}



\begin{document}
\frame{\titlepage}

\begin{frame}{Qué es un servidor}
\begin{itemize}
\item Es un ordenador que ofrece servicio
\item Aloja aplicaciones que ofrece un servicio
\item Las aplicaciones atienden peticiones de unos clientes a través de una red
\item Debería tener disponibilidad 24/7
\item Debería disponer de una buena velocidad de subida
\item ¿Alternativas? Servidor propio, servidor contratado, servidor en la nube\ldots
\end{itemize}
\end{frame}

\begin{frame}{Para qué queremos un servidor}
\begin{itemize}
\item Para ofrecernos toda clase de servicios a través de Internet
\item Disponibilidad de herramientas e información en cualquier lugar del mundo
\item ¿Ejemplos? Almacenamiento de ficheros, música y video en streaming, wikis, blogs, respositorios de versiones, correo\ldots
\end{itemize}
\end{frame}

\begin{frame}{Nota sobre seguridad}
Esta es una guía sencilla de instalación de un servidor que no cubre aspectos sofisticados de seguridad así que os recomendamos no tener información vital expuesta a Internet ni utilizar contraseñas que ya esteis usando. De cualquier modo, pos si alguien es un poco paranoico, el servidor puede asegurarse con herramientas como:
\begin{itemize}
\item El sentido común
\item mod\_security
\item aide
\item https
\end{itemize}
\end{frame}

\begin{frame}{Qué son los DNS}
\begin{itemize}
\item Internet comunica millones de dispositivos por todo el mundo. ¿Cómo nos referimos a uno en particular?
\item La respuesta es con números. Cada dispositivo (casi), un número. Los números se llaman {\bf direcciones IP} (i.e. 243.145.14.30)
\item Pero un número es difícil de recordar, mejor usar nombres (i.e. \url{www.gulucm.org})
\item Los DNS son {\bf servidores de nombres}. El SO se encarga de traducir los nombres a direcciones IP preguntando a los servidores de nombres.
\item Están jerarquizados y se pregunta del final hacia el principio. Primero se pregunta a un servidor maestro por 'org'. Luego a 'org' se le pregunta por 'gulucm' y a este último se le pregunta por 'www'.
\end{itemize}
\end{frame}

\begin{frame}{Dyndns/freedns}
\end{frame}
\begin{frame}{LAMP}
\end{frame}
\begin{frame}{Linux}
\end{frame}
\begin{frame}{Apache}
\end{frame}
\begin{frame}{Mysql}
\end{frame}
\begin{frame}{Php}
\end{frame}
\begin{frame}{Phpmyadmin}
\end{frame}
\begin{frame}{Drupal}
\end{frame}
\begin{frame}{Dokuwiki}
\end{frame}
\begin{frame}{FTP/Filezilla}
\end{frame}
\begin{frame}{SSH}
\end{frame}
\begin{frame}{SFTP/FireFTP}
\end{frame}
\begin{frame}{Samba}
\end{frame}
\begin{frame}{Plugins Drupal/Dokuwiki}
\end{frame}
\begin{frame}{Dokuwiki - ODT, PDF}
\end{frame}
\begin{frame}{Drupal - FCKEditor, SPAM}
\end{frame}
\begin{frame}{Banshee}
\end{frame}
\begin{frame}{NFS}
\end{frame}
\end{document}
