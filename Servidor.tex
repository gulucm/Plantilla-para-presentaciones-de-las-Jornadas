%\documentclass[spanish,10pt]{beamer}
\documentclass[spanish]{beamer}
% Class options include: notes, notesonly, handout, trans,
    %                        hidesubsections, shadesubsections,
    %                        inrow, blue, red, grey, brown
    \mode<presentation>
{
    \usetheme{Warsaw}
    \usecolortheme{rose}
    %\usecolortheme{dolphin}
    \setbeamertemplate{itemize items}[circle] % if you want a ball
    \setbeamertemplate{itemize subitem}[triangle] % if you wnat a circle
    \setbeamertemplate{itemize subsubitem}[ball] % if you want a triangle
    \setbeamertemplate{sections/subsections in toc}[circle]
    \setbeamertemplate{enumerate items}[circle]
    \setbeamercovered{transparent}
    % or whatever (possibly just delete it)
}
% Theme for beamer presentation.
\usepackage[spanish]{babel}
\usepackage[utf8]{inputenc}
%\usepackage{beamerthemesplit}

\mode<presentation>{
\usebackgroundtemplate{\includegraphics[width=\paperwidth]{gul_logo_b.png}}
} 

\usefonttheme{professionalfonts}% font de LaTeX
\usetheme{Warsaw}
% Tema escogido en este ejemplo
%\setBeamercovered{transparent}
%%%% packages y comandos personales %%%%
\usepackage[utf8]{inputenc}
\usepackage{latexsym} % Simbolos
\newtheorem{Ejemplo}{Ejemplo}
\newtheorem{Definicion}{Definicion}
\title[Monta tu propio servidor - GUL-UCM]
{Monta tu propio servidor
}
\author[\url{http://gulucm.org}]
{
Federico Mon\\
Salvador de la Puente
}
%{GUL-UCM} \\
%{\url{http://gulucm.org}}\\
%\vspace*{0.5cm}}
\institute{Facultad de Informática.\\Universidad Complutense de Madrid.}
\date{
I Jornadas de Software Libre - GUL-UCM\\
Marzo 2011}

\AtBeginSection[]
{
%%  \logo{\vbox{\vskip0.3cm\hbox{\includegraphics[width=2cm]{gulucm.eps}}}}
    \begin{frame}<beamer>
    \tableofcontents[currentsection]
    \end{frame}
\logo{}
}

%%\logo{\vbox{\vskip0.3cm\hbox{\includegraphics[width=2cm]{gulucm.eps}}}}

\newcommand*\oldmacro{}%
\let\oldmacro\insertshorttitle%
\renewcommand*\insertshorttitle{%
  \oldmacro\hfill%
  \insertframenumber\,/\,\inserttotalframenumber}



\begin{document}
\frame{\titlepage}

\begin{frame}{Qué es un servidor}
\begin{itemize}
\item Es un ordenador que ofrece servicio
\item Aloja aplicaciones que ofrece un servicio
\item Las aplicaciones atienden peticiones de unos clientes a través de una red
\item Debería tener disponibilidad 24/7
\item Debería disponer de una buena velocidad de subida
\item ¿Alternativas? Servidor propio, servidor contratado, servidor en la nube\ldots
\end{itemize}
\end{frame}

\begin{frame}{Para qué queremos un servidor}
\begin{itemize}
\item Para ofrecernos toda clase de servicios a través de Internet
\item Disponibilidad de herramientas e información en cualquier lugar del mundo
\item ¿Ejemplos? Almacenamiento de ficheros, música y video en streaming, wikis, blogs, respositorios de versiones, correo\ldots
\end{itemize}
\end{frame}

\begin{frame}{Nota sobre seguridad}
Esta es una guía sencilla de instalación de un servidor que no cubre aspectos sofisticados de seguridad así que os recomendamos no tener información vital expuesta a Internet ni utilizar contraseñas que ya esteis usando. De cualquier modo, pos si alguien es un poco paranoico, el servidor puede asegurarse con herramientas como:
\begin{itemize}
\item El sentido común
\item mod\_security
\item aide
\item https
\end{itemize}
\end{frame}

\begin{frame}{Qué son los DNS}
\begin{itemize}
\item Internet comunica millones de dispositivos por todo el mundo. ¿Cómo nos referimos a uno en particular?
\item La respuesta es con números. Cada dispositivo (casi), un número. Los números se llaman {\bf direcciones IP} (i.e. 243.145.14.30)
\item Pero un número es difícil de recordar, mejor usar nombres (i.e. \url{www.gulucm.org})
\item Los DNS son {\bf servidores de nombres}. El SO se encarga de traducir los nombres a direcciones IP preguntando a los servidores de nombres.
\item Están jerarquizados y se pregunta del final hacia el principio. Primero se pregunta a un servidor maestro por 'org'. Luego a 'org' se le pregunta por 'gulucm' y a este último se le pregunta por 'www'.
\end{itemize}
\end{frame}

\begin{frame}{Dyndns/freedns}
\begin{itemize}
\item Necesitamos algo que actualice las DNS con nuestras cambiantes IP 
dinámicas.
\item DynDNS y FreeDNS son servicios que nos permiten esto.
\item Nos creamos una cuenta en DynDNS, por ejemplo: \url{http://dyndns.com}
\item E instalamos \textbf{ddclient} en nuestro servidor
\item \textbf{ddclient} se encarga de actualizar los registros de DynDNS
cuando es necesario
\end{itemize}
\end{frame}
\begin{frame}{LAMP}
\begin{itemize}
\item \textbf{LAMP} son las siglas de \textbf{L}inux, \textbf{A}pache, \textbf{M}ySQL y \textbf{P}HP
\item Es una pila de tecnologías muy usadas conjuntamente para hacer sitios web.
\item También existen WAMP, MAMP, SAMP, OAMP y XAMPP
\item Se puede cambiar también PHP por Perl o Python
\item Además es lámpara en inglés
\end{itemize}
\end{frame}
\begin{frame}{Linux}
\begin{itemize}
\item Es el sistema operativo más usado para servidores web
\item Es libre, robusto y gratuito
\item Linux: \url{http://kernel.org}
\item GNU: \url{http://gnu.org}
\item GNU/Linux Debian: \url{http://debian.org}
\item GNU/Linux Ubuntu: \url{http://ubuntu.com}
\end{itemize}
\end{frame}
\begin{frame}{Apache}
\begin{itemize}
\item Es el servidor web más usado
\item Es libre, robusto y gratuito (A que parece Copy\&Paste?)
\item Es quizá la parte más importante del servidor
\item Apache: \url{http://apache.org}
\end{itemize}
\end{frame}
\begin{frame}{Mysql}
\begin{itemize}
\item Es el servidor de bases de datos libre más famoso
\item No hace falta que diga que es libre, gratuito y robusto, ¿verdad?
\item Es una parte esencial para tener muchos datos ordenadamente
\item Ha cambiado de manos recientemente, era de Sun y ahora de Oracle
\item MySQL: \url{http://mysql.org}
\end{itemize}
\end{frame}
\begin{frame}{Php}
\begin{itemize}
\item Es un lenguaje que permite hacer dinámicas las páginas web
\item Trabaja en conjunto con Apache insertando variables entre el HTML
\item Es uno de los lenguajes más usados en desarrollo web
\item Ha cambiado de manos recientemente, era de Sun y ahora de Oracle
\item PHP: \url{http://php.net}
\end{itemize}
\end{frame}
\begin{frame}{Phpmyadmin}
\end{frame}
\begin{frame}{Drupal}
\end{frame}
\begin{frame}{Dokuwiki}
\end{frame}
\begin{frame}{FTP/Filezilla}
\begin{itemize}
\item FTP significa File Transfer Protocol
\item Como bien se intuye, es un protocolo que permite transferir archivos
\item Donde más se usa es probablemente, para subir ``páginas web''
\item Necesitamos el demonio ftp en el servidor, y un cliente para conectarnos
\item Como cliente veremos Filezilla: \url{http://filezilla-project.org}
\end{itemize}
\end{frame}
\begin{frame}{SSH}
\begin{itemize}
\item SSH viene de Secure Shell
\item 
\item 
\end{itemize}
\end{frame}
\begin{frame}{SFTP/FireFTP}
\end{frame}
\begin{frame}{Samba}
\end{frame}
\begin{frame}{Plugins Drupal/Dokuwiki}
\end{frame}
\begin{frame}{Dokuwiki - ODT, PDF}
\end{frame}
\begin{frame}{Drupal - FCKEditor, SPAM}
\end{frame}
\begin{frame}{Banshee}
\end{frame}
\begin{frame}{NFS}
\end{frame}
\end{document}
