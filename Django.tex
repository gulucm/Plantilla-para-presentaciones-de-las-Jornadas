%\documentclass[spanish,10pt]{beamer}
\documentclass[spanish]{beamer}
% Class options include: notes, notesonly, handout, trans,
    %                        hidesubsections, shadesubsections,
    %                        inrow, blue, red, grey, brown
    \mode<presentation>
{
    \usetheme{Warsaw}
    \usecolortheme{rose}
    %\usecolortheme{dolphin}
    \setbeamertemplate{itemize items}[circle] % if you want a ball
    \setbeamertemplate{itemize subitem}[triangle] % if you wnat a circle
    \setbeamertemplate{itemize subsubitem}[ball] % if you want a triangle
    \setbeamertemplate{sections/subsections in toc}[circle]
    \setbeamertemplate{enumerate items}[circle]
    \setbeamercovered{transparent}
    % or whatever (possibly just delete it)
}
% Theme for beamer presentation.
\usepackage[spanish]{babel}
\usepackage[utf8]{inputenc}
%\usepackage{beamerthemesplit}

\mode<presentation>{\usebackgroundtemplate{\includegraphics[width=\paperwidth]{gul_logo_b.png}}} 

\usefonttheme{professionalfonts}% font de LaTeX
\usetheme{Warsaw}
% Tema escogido en este ejemplo
%\setBeamercovered{transparent}
%%%% packages y comandos personales %%%%
\usepackage[utf8]{inputenc}
\usepackage{latexsym} % Simbolos
\newtheorem{Ejemplo}{Ejemplo}
\newtheorem{Definicion}{Definicion}
\title[Django - GUL-UCM]
{Django
}
\author[\url{http://gulucm.org}]
{Salvador de la Puente González}
%{GUL-UCM} \\
%{\url{http://gulucm.org}}\\
%\vspace*{0.5cm}}
\institute{Facultad de Informática.\\Universidad Complutense de Madrid.}
\date{
I Jornadas de Software Libre - GUL-UCM\\
Marzo 2011}

\AtBeginSection[]
{
%%  \logo{\vbox{\vskip0.3cm\hbox{\includegraphics[width=2cm]{gulucm.eps}}}}
    \begin{frame}<beamer>
    \tableofcontents[currentsection]
    \end{frame}
\logo{}
}

%%\logo{\vbox{\vskip0.3cm\hbox{\includegraphics[width=2cm]{gulucm.eps}}}}

\newcommand*\oldmacro{}%
\let\oldmacro\insertshorttitle%
\renewcommand*\insertshorttitle{%
  \oldmacro\hfill%
  \insertframenumber\,/\,\inserttotalframenumber}



\begin{document}
\frame{\titlepage}

\section{Qué es Django}
\begin{frame}{Framework para aplicaciones web}
\begin{itemize}
\item No es un gestor de contenido, ¡es un framework!
\item Utiliza Python 2 (\url{http://www.python.org/})
\item Puede encontrarse en \url{http://www.djangoproject.com/}
\item Nombrado por Django Reinhardt\footnote{\url{http://es.wikipedia.org/wiki/Django\_Reinhardt}} (músico de Jazz) 
\end{itemize}
\end{frame}

\begin{frame}{Principios de diseño}
\begin{itemize}
\item DRY: Don't Repeat Yourself! (¡no te repitas!)
\item MVC: Model-View-Controller (pero con otros nombres\footnote{Más información en \url{http://docs.djangoproject.com/en/dev/faq/general/#django-appears-to-be-a-mvc-framework-but-you-call-the-controller-the-view-and-the-view-the-template-how-come-you-don-t-use-the-standard-names}}: Model-Template-View)
\item ¿Qué es el controlador? ¡El framework en sí!
\end{itemize}
\end{frame}

\section{Principales aportaciones}
\begin{frame}{ORM}
\begin{itemize}
\item ORM: Object-Relational Mapper (permite identificar tablas con objetos)
\item No soporta cambios en los modelos... ¡pero tenemos South!
\end{itemize}
\end{frame}

\begin{frame}{Aplicación admin}
\begin{itemize}
\item Permite generar una interfaz de administración para nuestros modelos
\item Es increiblemente configurable y ¡extensible!
\item Nos permite centrarnos en la experiencia de usuario
\end{itemize}
\end{frame}

\begin{frame}{Aplicación south}
\begin{itemize}
\item No es parte de Django pero es una aplicación para Django
\item Crea snapshots de nuestros modelos y permite cambiar entre ellos
\item A esto se le llama migración de esquemas (schema migration)
\item Se puede instalar mediante easy\_install o desde: \url{http://south.aeracode.org/}
\end{itemize}
\end{frame}

\begin{frame}{Otras características}
\begin{itemize}
\item Elegante diseño de urls: /archive/2005/11/30/noticia-ultima-hora (2005, 11, 30 y noticia-ultima-hora son parámetros)
\item Sistema de plantillas por texto: versátil y extensible
\item Sistema de caché
\item Internacionalización
\end{itemize}
\end{frame}

\section{El tutorial rápido}
\begin{frame}{Parte uno}
En \url{http://unoyunodiez.wordpress.com/2011/05/03/django-en-cero-coma-i/}
\begin{enumerate}
\item Configuración inicial
\item Sincronización de la base de datos
\item Probando el proyecto
\item Añadir una nueva aplicación
\item El modelo de la aplicación posts
\item Sincronizar la base de datos, el primer snapshot
\item Registrando el modelo en el panel de administración
\item El panel de administración autogenerado para posts
\item Vistas en Django
\item Plantillas en Django
\end{enumerate}
\end{frame}

\begin{frame}{Parte dos}
En \url{http://unoyunodiez.wordpress.com/2011/05/03/django-en-cero-coma-ii/}
\begin{enumerate}
\item Modificando el modelo 
\item Mejorando la administración de posts
\item Un sistema de comentarios
\item Gestión de posts y comentarios
\item Añadiendo la posibilidad de buscar
\end{enumerate}
\end{frame}

\end{document}
