%\documentclass[spanish,10pt]{beamer}
\documentclass[spanish]{beamer}
% Class options include: notes, notesonly, handout, trans,
    %                        hidesubsections, shadesubsections,
    %                        inrow, blue, red, grey, brown
    \mode<presentation>
{
    \usetheme{Warsaw}
    \usecolortheme{rose}
    %\usecolortheme{dolphin}
    \setbeamertemplate{itemize items}[circle] % if you want a ball
    \setbeamertemplate{itemize subitem}[triangle] % if you wnat a circle
    \setbeamertemplate{itemize subsubitem}[ball] % if you want a triangle
    \setbeamertemplate{sections/subsections in toc}[circle]
    \setbeamertemplate{enumerate items}[circle]
    \setbeamercovered{transparent}
    % or whatever (possibly just delete it)
}
% Theme for beamer presentation.
\usepackage[spanish]{babel}
\usepackage[utf8]{inputenc}
\usepackage{listings}
%\usepackage{beamerthemesplit}

\mode<presentation>{
\usebackgroundtemplate{\includegraphics[width=\paperwidth]{gul_logo_b.png}}
} 

\usefonttheme{professionalfonts}% font de LaTeX
\usetheme{Warsaw}
% Tema escogido en este ejemplo
%\setBeamercovered{transparent}
%%%% packages y comandos personales %%%%
\usepackage[utf8]{inputenc}
\usepackage{latexsym} % Simbolos
\newtheorem{Ejemplo}{Ejemplo}
\newtheorem{Definicion}{Definicion}
\title[Repositorio de versiones- GUL-UCM]
{Repositorio de versiones
}
\author[\url{http://gulucm.org}]
{
Federico Mon \\
Nehuén Eloy Benítez
}
%{GUL-UCM} \\
%{\url{http://gulucm.org}}\\
%\vspace*{0.5cm}}
\institute{Facultad de Informática.\\Universidad Complutense de Madrid.}
\date{
I Jornadas de Software Libre - GUL-UCM\\
Marzo 2011}

\AtBeginSection[]
{
%%  \logo{\vbox{\vskip0.3cm\hbox{\includegraphics[width=2cm]{gulucm.eps}}}}
    \begin{frame}<beamer>
    \tableofcontents[currentsection]
    \end{frame}
\logo{}
}

%%\logo{\vbox{\vskip0.3cm\hbox{\includegraphics[width=2cm]{gulucm.eps}}}}

\newcommand*\oldmacro{}%
\let\oldmacro\insertshorttitle%
\renewcommand*\insertshorttitle{%
  \oldmacro\hfill%
  \insertframenumber\,/\,\inserttotalframenumber}



\begin{document}
    \frame{\titlepage}
    \section{Control de versiones}
        \begin{frame}{Las versiones de un fichero}
         \begin{itemize}
	\item ¿Qué es una versión de algo?
	\item Hitos
	\item Backups
	\item Trabajar en equipo
         \end{itemize}
       \end{frame} 


        \begin{frame}{Sistemas de Control de Versiones}
         \begin{itemize}
       \item Los sistemas de control de versiones hacen más fácil controlar las 
        versiones de nuestro trabajo.
       \item Permiten:
         \begin{itemize}
	\item Hacer un comentario para cada versión que queremos
         guardar,
        \item Llevar un control de "quién ha hecho qué"
	\item Volver atrás en las versiones
        \item Facilitan que todas las personas involucradas 
        puedan acceder siempre a la última versión del trabajo
	\item Ver el historial de las versiones
	\item Ver las diferencias entre éstas.
       \end{itemize} 
       \end{itemize} 
       \end{frame} 
        
        \begin{frame}{Central vs. distribuido}
        Existen dos tipos de sistemas de control de versiones, los centralizados
        , donde toda la información es almacenada en un servidor central, 
        llamado repositorio, y los distribuidos, donde esta información se 
        replica también en los clientes.
        
        La ventaja más clara de los sistemas distribuidos frente a los 
        centralizados es que no es necesario poder conectar con el servidor 
        central para cada acción que queremos realizar.
       \end{frame} 
       
       \begin{frame}{Conceptos básicos}
       \begin{description}[Left]
               \item [Repositorio:] Servidor donde hay una copia del código y
                 de toda la información extra para llevar el control de 
                 versiones
               \item [Working copy:] Copia del código donde se trabaja (local) 
               \item [Checkout:] Creación del Working copy
               \item [Commit:] Subir cambios al repositorio
                \item [Update:] Actualizar el working copy
            
       \end{description}
       \end{frame}
       
       
       \begin{frame}{Más conceptos básicos}
       \begin{description}[Left]
           \item [Revision:] Versión, Foto del working copy
           \item [Parent:] Versión de donde viene otra
           \item [First revision:] Versión inicial, no tiene padres
           \item [Branches:] Versiones diferentes con el mismo padres
           \item [Merge revision:] Versión que tiene dos padres (unifica)
           \item [Head:] Versión que no tiene hijos (última versión?)
          \end{description}
       \end{frame}


        \section{Mercurial}
        
        \begin{frame}{Introducción}
            \begin{itemize}
                \item Tan simple como SVN
                \item Tan potente como Git
                \item No necesitamos red, ni pendrives
                \item Nos permite llevar un registro local
		\item Permite usar branches y tags
            \end{itemize}
        \end{frame}
        
        
        \begin{frame}{Crear un repositorio}
            \begin{itemize}
                \item hg init
                \item hg add *
                \item hg serve
            \end{itemize}
        \end{frame}
        
        
        \begin{frame}{Copiarse un repositorio}
            \begin{itemize}
                \item hg clone /media/pendrive/projecto miprojecto
                \item hg clone http://localhost:8000/ miprojecto
            \end{itemize}
        \end{frame}
        
       
        \begin{frame}{Subir y bajar}
             \begin{itemize}
                \item hg push
                \item hg pull
             \end{itemize}
	\end{frame}
 
        \begin{frame}{Trabajar en local}
             \begin{itemize}
                \item hg add
                \item hg remove
                \item hg cp
                \item hg mv
             \end{itemize}
        \end{frame}
 
        \begin{frame}{Guardar en local}
             \begin{itemize}
                \item hg commit
                \item hg revert
                \item hg update
             \end{itemize}
        \end{frame}
 
	\begin{frame}{Interfaces gráficas}
             \begin{itemize}
                \item EasyHg\url{https://code.soundsoftware.ac.uk/projects/easyhg/files}
                \item TortoiseHg\url{http://bitbucket.org/tortoisehg/stable/wiki/Home}
                \item MercurialEclipse: \url{http://javaforge.com/project/HGE}
             \end{itemize}
		
	\end{frame}
	
	\begin{frame}{Trabajando online}
		Opción 1: BitBucket: \url{https://bitbucket.org/}
	\end{frame}
               
	\begin{frame}{Trabajando online}
		Opción 2: Dropbox: \url{https://dropbox.com/}
		\begin{itemize}
			\item hg clone ~/myproj ~/Dropbox/myproj-hg --noupdate
			\item vim ~/myproj/.hg/hgrc
			\item [paths]
			\item default = /home/sgraham/Dropbox/myproj-hg/
		\end{itemize}
	\end{frame}
       
    \begin{frame}{Referencias}
	\begin{itemize}
	\item \url{http://www.h4ck3r.net/2010/05/11/mercurial-hg-with-dropbox/}
	\item \url{http://mercurial.selenic.com/wiki/OtherTools}
	\item \url{https://code.soundsoftware.ac.uk/projects/easyhg/files}
	\item \url{http://javaforge.com/project/HGE}
	\end{itemize}
        \end{frame}


 

%%%%%%%%%%%%%%%% A partir de aquí sólo leer para aprender del pasado        
%            \begin{itemize}
%                \item Asociación de estudiantes de la Universidad Complutense.
%                \item Sede en la Facultad de Informática.
%                \item Dedicada a promover el uso de Software Libre.
%            \end{itemize}
%        \end{frame}
%        \begin{frame}{}
%            Se trata de una filosofía que queda resumida y representada en 4 
%            libertades:
%            \begin{enumerate}
%            \setcounter{enumi}{-1}
%                \item La libertad de usar el programa, con cualquier propósito.
%                \item La libertad de estudiar \footnote{Las libertades 1 y 3 
%                requieren acceso al código fuente porque estudiar y modificar 
%                software sin su código fuente es muy poco viable} cómo funciona 
%                el programa y modificarlo, adaptándolo a tus necesidades.
%                \item La libertad de distribuir copias del programa, con lo cual
%                 puedes ayudar a tu prójimo.
%                \item La libertad de mejorar el programa y hacer públicas esas 
%                mejoras a los demás, de modo que toda la comunidad se beneficie.
%            \end{enumerate}
%        \end{frame}
%
%    \section{Quiénes somos}
%        \begin{frame}{Representación}
%            \begin{description}[Left]
%                \item [Presidente:] Federico G. Mon Trotti
%                \item [Vicepresidente:] Nehuén Eloy Benítez
%                \item [Secretario:] Sergio D'Antonio
%                \item [Tesorero:] Luis San Juan
%                \item [Vocal de representación estudiantil:] Jorge Maestre Vidal
%                \item [Vocal de actividades:] Salvador de la Puente González
%            \end{description}
%        \end{frame}
%        \begin{frame}{Recursos y contacto}
%            (link a estatutos, lista de correo, pagina web, etc)
%        \end{frame}
\end{document}
