%\documentclass[spanish,10pt]{beamer}
\documentclass[spanish]{beamer}
% Class options include: notes, notesonly, handout, trans,
    %                        hidesubsections, shadesubsections,
    %                        inrow, blue, red, grey, brown
    \mode<presentation>
{
    \usetheme{Warsaw}
    \usecolortheme{rose}
    %\usecolortheme{dolphin}
    \setbeamertemplate{itemize items}[circle] % if you want a ball
    \setbeamertemplate{itemize subitem}[triangle] % if you wnat a circle
    \setbeamertemplate{itemize subsubitem}[ball] % if you want a triangle
    \setbeamertemplate{sections/subsections in toc}[circle]
    \setbeamertemplate{enumerate items}[circle]
    \setbeamercovered{transparent}
    % or whatever (possibly just delete it)
}
% Theme for beamer presentation.
\usepackage[spanish]{babel}
\usepackage[utf8]{inputenc}
\usepackage{listings}
%\usepackage{beamerthemesplit}

\mode<presentation>{
\usebackgroundtemplate{\includegraphics[width=\paperwidth]{gul_logo_b.png}}
} 

\usefonttheme{professionalfonts}% font de LaTeX
\usetheme{Warsaw}
% Tema escogido en este ejemplo
%\setBeamercovered{transparent}
%%%% packages y comandos personales %%%%
\usepackage[utf8]{inputenc}
\usepackage{latexsym} % Simbolos
\newtheorem{Ejemplo}{Ejemplo}
\newtheorem{Definicion}{Definicion}
\title[Git y Mercurial vs. SVN - GUL-UCM]
{Git y Mercurial vs. Subversion
}
\author[\url{http://gulucm.org}]
{
Federico Mon \\
Nehuén Eloy Benítez
}
%{GUL-UCM} \\
%{\url{http://gulucm.org}}\\
%\vspace*{0.5cm}}
\institute{Facultad de Informática.\\Universidad Complutense de Madrid.}
\date{
I Jornadas de Software Libre - GUL-UCM\\
Marzo 2011}

\AtBeginSection[]
{
%%  \logo{\vbox{\vskip0.3cm\hbox{\includegraphics[width=2cm]{gulucm.eps}}}}
    \begin{frame}<beamer>
    \tableofcontents[currentsection]
    \end{frame}
\logo{}
}

%%\logo{\vbox{\vskip0.3cm\hbox{\includegraphics[width=2cm]{gulucm.eps}}}}

\newcommand*\oldmacro{}%
\let\oldmacro\insertshorttitle%
\renewcommand*\insertshorttitle{%
  \oldmacro\hfill%
  \insertframenumber\,/\,\inserttotalframenumber}



\begin{document}
    \frame{\titlepage}
    \section{Control de versiones}
        \begin{frame}{Las versiones de un fichero}
        Qué es una versión de algo?
        Todos cuando trabajamos con un ordenador en un proyecto, llegamos a un
        punto en el cual sucede un hito.
        Este hito depende de en qué estemos trabajando: pero puede ser que 
        tengamos un documento aceptable, o que hayamos solucionado un problema 
        parcialmente... la cuestión es que queremos guardar nuestro trabajo, 
        pero debemos seguir trabajando de todas maneras.
        
        Normalmente con un ordenador, lo que hacemos es guardar el fichero con 
        un nombre y ponerle un número, y seguimos trabajando sobre una copia de
        este fichero.
        
        El problema es que cuando el número de versiones empieza a crecer, es 
        muy difícil llevar manualmente un seguimiento de las versiones, volver 
        a la que queremos, etc.
        
        Esta tarea se complica mucho más cuando son varias personas las que 
        trabajan en un proyecto en paralelo, y además juntar los cambios hechos 
        es algo inevitable.
       \end{frame} 
        \begin{frame}{Sistemas de Control de Versiones}
        
        Los sistemas de control de versiones hacen más fácil controlar las 
        versiones de nuestro trabajo.
        
        Permiten por ejemplo, hacer un comentario para cada versión que queremos
         guardar, llevan un control de "quién ha hecho qué", permiten volver 
        atrás en las versiones, y facilitan que todas las personas involucradas 
        puedan acceder siempre a la última versión del trabajo, así como ver el 
        historial de las versiones, o las diferencias entre éstas.
        
        Existen dos tipos de sistemas de control de versiones, los centralizados
        , donde toda la información es almacenada en un servidor central, 
        llamado repositorio, y los distribuidos, donde esta información se 
        replica también en los clientes.
        
        La ventaja más clara de los sistemas distribuidos frente a los 
        centralizados es que no es necesario poder conectar con el servidor 
        central para cada acción que queremos realizar.
       \end{frame} 
       
       \begin{frame}{Conceptos básicos}
       \begin{description}[Left]
               \item [Repositorio:] Servidor donde hay una copia del código y
                 de toda la información extra para llevar el control de 
                 versiones
               \item [Working copy:] Copia del código donde se trabaja (local) 
               \item [Checkout:] Creación del Working copy
               \item [Commit:] Subir cambios al repositorio
                \item [Update:] Actualizar el working copy
            
       \end{description}
       \end{frame}
       
       
       \begin{frame}{Más conceptos básicos}
       \begin{description}[Left]
           \item [Revision:] Versión, Foto del working copy
           \item [Parent:] Versión de donde viene otra
           \item [First revision:] Versión inicial, no tiene padres
           \item [Branches:] Versiones diferentes con el mismo padres
           \item [Merge revision:] Versión que tiene dos padres (unifica)
           \item [Head:] Versión que no tiene hijos (última versión?)
          \end{description}
       \end{frame}
       
       
        \section{CVCS: Subversion}
         
        
        
       \begin{frame}{Subversion}
       \begin{itemize}
        \item Subversion es una evolución de CVS.
        \item Consta de un repositorio centralizado, y tantas working copies 
         como sean necesarios, que dependen directamente de éste.
        \end{itemize}
          \end{frame}
\begin{subsection}{Uso básico de svn}
\begin{frame}[fragile]
\frametitle{Montar un servidor SVN}
%\lstinputlisting[language=bash]{input.txt}
\begin{lstlisting}
$ svnadmin create /svnrepos
$ vim /svnrepos/conf/svnserve.conf
anon-access = none // read
auth-access = write
password-db = passwd
$ vim svnrepos/conf/passwd
tony = mypassword
$ svn import /projects/myproject 
file:///svnrepos/myproject
$ svnserve -d
$ svn co 
svn://localhost/svnrepos/myyrailsproject
\end{lstlisting}
\end{frame}


\begin{frame}[fragile]
\frametitle{Checkout inicial}
\begin{lstlisting}
$ svn checkout http://svn.collab.net/repos/svn/trunk
A    trunk/Makefile.in
A    trunk/ac-helpers
A    trunk/ac-helpers/install.sh
A    trunk/ac-helpers/install-sh
A    trunk/build.conf
...
Checked out revision 8810.
\end{lstlisting}       
\end{frame}



\begin{frame}[fragile]
\frametitle{Resto de actualizaciones}
\begin{itemize}
\item svn update
\end{itemize}
\end{frame}


\begin{frame}[fragile]
\frametitle{Trabajamos y actualizamos nuestro working copy}
\begin{itemize}
\item svn add
\item svn delete
\item svn copy
\item svn move
\end{itemize}
\end{frame}


\begin{frame}[fragile]
\frametitle{Examinamos nuestro trabajo}
\begin{itemize}
\item svn status
\item svn diff
\end{itemize}
\end{frame}

\begin{frame}[fragile]
\frametitle{Deshacer cambios}
\begin{itemize}
\item svn revert
\end{itemize}
\end{frame}


\begin{frame}[fragile]
\frametitle{Actualizamos nuestro WC}
\begin{itemize}
\item svn update
\item svn resolve
\end{itemize}
\end{frame}

\begin{frame}[fragile]
\frametitle{Subimos nuestros cambios}
\begin{itemize}
\item svn commit
\end{itemize}
\end{frame}
\end{subsection}


        \begin{frame}{Problemas en Subversion}
           \begin{itemize}
              \item Commit y Update
              \item Branches
           \end{itemize}
	\end{frame}
        \section{DVCS: Git y Mercurial}
%%%%%%%%%%%%%%%%%%%%%%%%%%%%%%%%%%%%%%
%     Aquí iría GIT
%%%%%%%%%%%%%%%%%%%%%%%%%%%%%%%%%%%%%%        

%%%%%%%%%%%%%%%%%%%%%%%%%%%%%%%%%%%%%%
%     Aquí iría Mercurial
%%%%%%%%%%%%%%%%%%%%%%%%%%%%%%%%%%%%%5

        \section{Mercurial}
        
        \begin{frame}{Introducción}
            \begin{itemize}
                \item Tan simple como SVN
                \item Tan potente como Git
                \item No necesitamos red
            \end{itemize}
        \end{frame}
        
        
        \begin{frame}{Crear y servir un repo}
            \begin{itemize}
                \item hg init
                \item hg serve
            \end{itemize}
        \end{frame}
        
        
        \begin{frame}{Clonar un repo}
            \begin{itemize}
                \item hg clone http://localhost:8000/ what      
            \end{itemize}
        \end{frame}
        
        
        \begin{frame}{Diferencias con git}
            \begin{itemize}
                \item Mercurial no permite Octopus merges
                \item Branches
                \begin{itemize}
                    \item Hg: un branch está empotrada en un commit
                    \item Git: un branch es un puntero a un commit
                \end{itemize}
                \item \url{http://www.wikivs.com/wiki/Git_vs_Mercurial}
            \end{itemize}
        \end{frame}
        
        
        
               
        \begin{frame}{Usar Git en Mercurial y viceversa} 
        Hg to Git: \url{http://hg-git.github.com/}
        Git to Hg: \url{https://github.com/offbytwo/git-hg}
        \end{frame}
	\begin{frame}{Conclusiones}
	    \begin{itemize}
	    \item \url{http://gitvsmercurial.com/}
	    \item Git si necesitas algo muy flexible
	    \item Mercurial si necesitas algo normal
        \end{itemize}
        \end{frame}
       
    \begin{frame}{Referencias}
     \begin{itemize}
	    \item \url{http://www.russellbeattie.com/blog/distributed-revision-control-systems-git-vs-mercurial-vs-svn}
	    \item \url{https://importantshock.wordpress.com/2008/08/07/git-vs-mercurial/}
	    \item \url{http://www.wikivs.com/wiki/Git_vs_Mercurial}
	    \item \url{https://felipec.wordpress.com/2011/01/16/mercurial-vs-git-its-all-in-the-branches/}
        \item \url{http://stackoverflow.com/questions/35837/what-is-the-difference-between-mercurial-and-git}
\end{itemize}
        \end{frame}


 

%%%%%%%%%%%%%%%% A partir de aquí sólo leer para aprender del pasado        
%            \begin{itemize}
%                \item Asociación de estudiantes de la Universidad Complutense.
%                \item Sede en la Facultad de Informática.
%                \item Dedicada a promover el uso de Software Libre.
%            \end{itemize}
%        \end{frame}
%        \begin{frame}{}
%            Se trata de una filosofía que queda resumida y representada en 4 
%            libertades:
%            \begin{enumerate}
%            \setcounter{enumi}{-1}
%                \item La libertad de usar el programa, con cualquier propósito.
%                \item La libertad de estudiar \footnote{Las libertades 1 y 3 
%                requieren acceso al código fuente porque estudiar y modificar 
%                software sin su código fuente es muy poco viable} cómo funciona 
%                el programa y modificarlo, adaptándolo a tus necesidades.
%                \item La libertad de distribuir copias del programa, con lo cual
%                 puedes ayudar a tu prójimo.
%                \item La libertad de mejorar el programa y hacer públicas esas 
%                mejoras a los demás, de modo que toda la comunidad se beneficie.
%            \end{enumerate}
%        \end{frame}
%
%    \section{Quiénes somos}
%        \begin{frame}{Representación}
%            \begin{description}[Left]
%                \item [Presidente:] Federico G. Mon Trotti
%                \item [Vicepresidente:] Nehuén Eloy Benítez
%                \item [Secretario:] Sergio D'Antonio
%                \item [Tesorero:] Luis San Juan
%                \item [Vocal de representación estudiantil:] Jorge Maestre Vidal
%                \item [Vocal de actividades:] Salvador de la Puente González
%            \end{description}
%        \end{frame}
%        \begin{frame}{Recursos y contacto}
%            (link a estatutos, lista de correo, pagina web, etc)
%        \end{frame}
\end{document}
